% -- Encoding UTF-8 without BOM
% -- XeLaTeX => PDF (BIBER)

\documentclass[]{cv-style}          % Add 'print' as an option into the square bracket to remove colours from this template for printing. 
                                    % Add 'espanol' as an option into the square bracket to change the date format of the Last Updated Text

\sethyphenation[variant=british]{english}{} % Add words between the {} to avoid them to be cut 

\usepackage{etoolbox}

\patchcmd{\header}
{anchor=center}
{anchor=center,text width=\textwidth,align=center}
{}{}

\patchcmd{\header}
{minimum height=2cm}
{minimum height=2.4cm}
{}{}

\patchcmd{\header}
{{\thinfont #1}{\bodyfont  #2}}
{{\thinfont #1}{\bodyfont  #2}\\[6pt]{\bodyfont\large PhD - Directeur Recherche et Développement}}
{}{}

\apptocmd{\header}{\vskip2\parskip}{}{}

\patchcmd{\aside}{(1, 1.87)}{(1, 2.4)}{}{}
\begin{document}

\header{Alexis}{Duque}           % Your name
\lastupdated

%----------------------------------------------------------------------------------------
%	SIDEBAR SECTION  -- In the aside, each new line forces a line break
%----------------------------------------------------------------------------------------

\begin{aside}
%
\section{Contact}
64 rue Lucette et René Desgrand
69100, Villeurbanne
France
~
+33 6 51 24 37 76
~
alexisd61@gmail.com
alexis.duque@rtone.fr
%
\section{Langues}
Anglais, professionnel
{\color{red} $\varheartsuit$} Espagnol, bilingue
Mandarin, notions
%
\section{Recherche}
{\color{red} $\varheartsuit$} Internet of Things
Machine Learning
Visible Light Comm.
{\color{red} $\varheartsuit$} Bluetooth LE
Software Defined Radio
Lightweight Crypto.
{\color{red} $\varheartsuit$} IoT Cybersecurity
Wireless Comm.
%
\section{Programmation}
{\color{red} $\varheartsuit$} C/C++, Golang
Python, Jupyter, TensorFlow, TensorFlow Lite
Docker, K8S, Kubeflow
{\color{red} $\varheartsuit$} Embedded Linux \& Android
ARM Cortex M MCUs
Android SDK/NDK
Java, Kotlin
SysAdmin, Agile
{\color{red} $\varheartsuit$} Git, Gitlab, CI/CD
\LaTeX{}
%
\end{aside}

%  \vspace{-0.2cm}

%----------------------------------------------------------------------------------------

%----------------------------------------------------------------------------------------
%	WORK EXPERIENCE SECTION
%----------------------------------------------------------------------------------------
 \vspace{0.15cm}

\section{Expériences Professionnelles}

\begin{entrylist}
\entry
  {Depuis 2017}
  {Rtone}
  {Lyon, France}
  {\jobtitle{Directeur R\&D}\\
  Gestion de projets collaboratifs : SME Instrument, Eurostars, ANR\\
  Projets en cours :
\begin{itemize}
    \item FUI PACLIDO (2017-20): Cryptographie légère pour l'IoT. Détection de cyberattaques en utilisant des techniques de Machine Learning
    \item H2020 SDR4IoT (2020-21): Fingerprinting et authentification d'objets connectés avec de la radio logicielle et du Machine Learning
    \item ANACONDA (2020-2022): Automatisation de la validation sécuritaire et des tests d'intrusions sur des objets connectés
\end{itemize}
Responsable technique de l'offre de conseil et audit de sécurité pour l'IoT\\
Enseignement : Système embarqués et IoT, Bluetooth, VLC, cybersécurité IoT\\
Speaker lors de conférences internationales : Devoxx (FR/BE/MA), FOSDEM, MixIT\\
Rédaction d'articles techniques et scientifiques
}
%------------------------------------------------
\entry
  {2015-2018}
  {Rtone}
  {Lyon, France}
  {\jobtitle{Doctorant}\\
  Sujet de thèse : Bidirectional Visible Light Communications for the Internet of Things\\ 
  Développement d'objets communicants grâce à la lumière visible et un smartphone\\
  Redaction et dépôt d'un brevet à l'INPI et l'EPO (European Patent Office)
 }
%------------------------------------------------
\entry
  {2014-2017}
  {Rtone}
  {Lyon, France}
  {\jobtitle{Ingénieur R\&D Systèmes Embarqué}\\
  Programmation sur micro-contrôleurs:
  \begin{itemize}
    \item Développement d'un Body-Area-Network pour un contexte sportif (802.15.4)
    \item Mode Low Power et optimisation de la consommation énergétique
    \item STM32. ARM Cortex M0+, M0, M3. TI MSP430
  \end{itemize}
  Développement de firmwares Bluetooth Low Energy sur SoC Nordic et Bluegiga.\\
  Conception d'une passerelle 3G Linux Embarqué: C++, Projet Yocto\\
  Applications mobiles :
  \begin{itemize}
    \item Expertise sur Bluetooth Low Energy et Ultrasons
    \item Android SDK et NDK. X-Plateforme avec Apache Cordova, Ionic, AngularJS
  \end{itemize}
  Applications Web :
  \begin{itemize}
    \item Évolutions et améliorations de plusieurs applications web Java dans le milieu de l'Internet des Objets (Spring, Hibernate, GWT, OSGI)
  \end{itemize}}

%------------------------------------------------
 \entry
  {S1 2015 }
  {ENTEL, Universidad Politecnica de Cataluna}
  {Barcelone, Espagne}
  {\jobtitle{Light sensor development for the Ara platform}\\
  Projet de recherche au sein de l'équipe Wireless Network Group, 5 mois :
  \begin{itemize}
    \item Etat de l'art et veille sur Visible Light Communications (VLC)
    \item Etude de la plate-forme de développement du smartphone modulaire Ara
    \item Conception du circuit électronique d'un récepteur pour la communication par modulation lumineuse. Développement C et Android (SDK et NDK)
  \end{itemize}}
\end{entrylist}

\begin{entrylist}
%------------------------------------------------
\entry
  {Depuis 2013}
  {Développeur Freelance}
  {Lyon, France}
  {\jobtitle{Développeur Web et Mobile}\\
  Développement d'applications web et mobiles en utilisant des technologies comme Java, Swift, Symphony2/3, Angular, Polymer\\
  Exemple de projets:
  \begin{itemize}
    \item Université Technologique de Troyes: développement d'une application Android d'aide à conduite de réunion
    \item Sanofi: développement d'un outils d'analyse statistique de résultats de test de lots de vaccins
    \item Michelin : conception de l’outil de reporting énergétique de l'usine de Troyes
    \item AMRC Europe : conception et développement d'une applications web de dimensionnement de systèmes de protection incendie en suivant les norme NF, NFPA et APSADR1
    \item 2S Medical : Développement d'un système de dossier médical électronique basé sur OpenMRS et Bahmni
  \end{itemize}}
%------------------------------------------------
\end{entrylist}

%----------------------------------------------------------------------------------------
%	EDUCATION SECTION
%----------------------------------------------------------------------------------------
%\newpage
\section{Formation}

\begin{entrylist}
%------------------------------------------------
\entry
{2015--2018}
{PhD. {\normalfont Bidirectional Visible Light Communications for the Internet of Things}}
{CITI Lab}
{\bodyfontit{Equipe INRIA AGORA, contrat CIFRE avec Rtone}}
%------------------------------------------------
\entry
{2009--2015}
{Ing. {\normalfont mention Télécommunications, Services et Usages.}}
{INSA de Lyon}
{\bodyfontit{Semestre ERASMUS à l'UPC-ETSETB, Barcelone, Espagne} \\
\bodyfontit{2 ans en section internationale Asie. Stage ouvrier de 2 mois à Shanghai, Chine}}
%------------------------------------------------
\end{entrylist}

%----------------------------------------------------------------------------------------
%	OTHER QUALIFICATIONS SECTION
%----------------------------------------------------------------------------------------

%\section{other qualifications}

%\begin{entrylist}
%------------------------------------------------
%\entry
%{2013}
%{Qualification}
%{Institution}
%{\vspace{-0.3cm}}
%------------------------------------------------
%\entry
%{2011}
%{Qualification}
%{Institution}
%{\vspace{-0.3cm}}
%------------------------------------------------
%\end{entrylist}

%----------------------------------------------------------------------------------------
%	AWARDS SECTION
%----------------------------------------------------------------------------------------

%\section{awards}

%\begin{entrylist}
%------------------------------------------------
%\entry
%{2014}
%{Google Summer of Code Student}
%{OpenMRS}
%{Open Source contributor at OpenMRS in the framework of Google Summer of Code fellowship. }
%------------------------------------------------
%\end{entrylist}
%----------------------------------------------------------------------------------------
%	PUBLICATION SECTION
\needspace{5\baselineskip}
\section{Publications}
 \vspace{-0.2cm}
\large{\textbf{Articles dans des conférences académiques}}

\normalsize
\begin{publist}
%------------------------------------------------
\pub
{A. Duque, A. Lahmadi, N. Heraief, and J. Francq}
{“MitM Attack Detection in BLE Networks using Reconstruction and Classification Machine Learning Techniques”}
{Proceedings of the 2nd Workshop on Machine Learning for Cybersecurity}
{MLCS’20, (Ghent, Belgium)}\\
\pub
{A. Duque, R. Stanica, H. Rivano, and A. Desportes}
{“Analytical and simulation tools for optical camera communications”}
{Elsevier Computer Communications}
{Vol. 120, pp. 52-62, July 2020}\\
\pub
{A. Duque, R. Stanica, H. Rivano, and A. Desportes}
{“Performance Evaluation of LED-to-Camera Communications”}
{Proceedings of the 22nd ACM International Conference on Modeling, Analysis and Simulation of Wireless and Mobile Systems}
{MSWiM’19, (Miami Beach, FL, USA)}\\
\pub
{A. Duque, R. Stanica, H. Rivano, C. Goursaud, and A. Desportes}
{“Poster: Insights into RGB-LED to Smartphone Communication”}
{Proceedings of the 2018 International Conference on Embedded Wireless Systems and Networks}
{EWSN’18, (Madrid, Spain)}\\
\pub
{A. Duque, R. Stanica, H. Rivano, and A. Desportes}
{“Decoding methods in LED-to-smartphone bidirectional communication for the IoT”}
{Proceedings of the 2018 Global LIFI Congress (GLC)}
{GLC’18, (Paris, France)}\\
\pub
{A. Duque, R. Stanica, H. Rivano, and A. Desportes}
{“SeedLight: Hardening LED-to-Camera Communication with Random Linear Coding”}
{Proceedings of the 4th Workshop on Visible Light Communication System}
{VLCS’17, (Snowbird, Utah, USA)}\\

\end{publist}

\begin{publist}
\pub
{A. Duque, R. Stanica, H. Rivano, and A. Desportes}
{“Demo : Off-the-shelf bi-directional visible light communication module for IoT devices and smartphones”}
{Proceedings of the 2017 International Conference on Embedded Wireless Systems and Networks}
{EWSN’17, (Uppsala, Sweden), 2017}\\
\pub
  {A. Duque, R. Stanica, H. Rivano, and A. Desportes}
  {“Unleashing the power of led- to-camera communications for iot devices”}
  {Proceedings of the 3rd Workshop on Visible Light Communication System}
  {VLCS’16, (New York, NY, USA)}\\
\pub
  {A. Duque, R. Stanica, H. Rivano, and A. Desportes} {“Demo: Unleashing the power of led- to-camera communications for iot devices”} {Proceedings of the 3rd Workshop on Visible Light Communication Systems} {VLCS’16, (New York, NY, USA)}
\end{publist}

%------------------------------------------------

\large{\textbf{Brevets}}

\normalsize
\begin{publist}
%------------------------------------------------
\pat
{A. Duque, A. Desportes,  R. Stanica, H. Rivano}
{2017}
{"Procédés de communication en lumière visible"}
{Brevet Européen N° EP18157382}
{Date de Priorité : 17 février 2017}
\\
\end{publist}
\large{\textbf{Presse technique}}

\normalsize
\begin{publist}
%------------------------------------------------
\pub
  {A. Duque} {“TensorFlow Lite im Vertical Farming”} {Linux Magazin } {Avril 2020}\\
\pub
  {A. Duque} {“Run TensorFlow models on edge devices”} {ADMIN Magazine} {Numéro 57, Mai 2020}\\
\pub
  {A. Duque} {“Machine Learning sur des objets connectés avec TensorFlow Lite pour l’agriculture verticale”} {Linux Magazine France} {Numéro 236, Juillet-Août 2020}
  
%------------------------------------------------
\end{publist}

% \large{\textbf{Présentations dans des conférences IT}}

% \normalsize
% \begin{publist}
% %------------------------------------------------
% \prez
% {A. Duque, A. Desportes,  R. Stanica, H. Rivano}
% {2017}
% {"Procédés de communication en lumière visible"}
% {Brevet Européen N° : EP18157382}
% {Date de Priorité : 17 février 2017}
% \\
% \end{publist}

%----------------------------------------------------------------------------------------
%	AWARDS SECTION
%----------------------------------------------------------------------------------------

\section{Certifications}

\begin{entrylist}
%------------------------------------------------
\entry
{Juin 2020}
{TensorFlow Developer Certificate}
{Google}
{ID 8TGQMHD85RS7 \\ Intégration d'algorithmes de Machine Learning dans les outils et les applications en production. Développement de modèles TensorFlow utilisant des réseaux de neurones convolutifs, pour du traitement du langage naturel, pour la vision par ordinateur, ou des séries temporelles.}
\entry
{Octobre 2020}
{Bioinformatics Specialization}
{UC San Diego}
{ID F847KGZXMN8F \\ Fondamentaux de biologie et développement d'outils informatiques et d'algorithmes appliqués à la résolution de problèmes courants en biologie moderne: séquençage de génome, identification de base génétique de maladie, recherche de molécules actives, etc.}
%------------------------------------------------
\end{entrylist}

%----------------------------------------------------------------------------------------
%	INTERESTS SECTION
%----------------------------------------------------------------------------------------

\section{Centres d'interêt}
  \vspace{-0.2cm}
  
\textbf{Professionnel :} veille technologique, sport sciences, contributeur open-source sur le projet OpenMRS, participation au Google Summer of Code 2014 et 2016 \\
\textbf{Sport :} duathlon, triathlon, athlétisme \\
\textbf{Culture :} lecture, pratique de la guitare
%----------------------------------------------------------------------------------------
\end{document}