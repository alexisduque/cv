% -- Encoding UTF-8 without BOM
% -- XeLaTeX => PDF (BIBER)

\documentclass[]{cv-style}          % Add 'print' as an option into the square bracket to remove colours from this template for printing. 
                                    % Add 'espanol' as an option into the square bracket to change the date format of the Last Updated Text

\sethyphenation[variant=british]{english}{} % Add words between the {} to avoid them to be cut 

\usepackage{etoolbox}

\patchcmd{\header}
{anchor=center}
{anchor=center,text width=\textwidth,align=center}
{}{}

\patchcmd{\header}
{minimum height=2cm}
{minimum height=2.4cm}
{}{}

\patchcmd{\header}
{{\thinfont #1}{\bodyfont  #2}}
{{\thinfont #1}{\bodyfont  #2}\\[6pt]{\bodyfont\large PhD - Responsable R\&D et Sécurité}}
{}{}

\apptocmd{\header}{\vskip2\parskip}{}{}

\patchcmd{\aside}{(1, 1.87)}{(1, 2.4)}{}{}
\begin{document}

\header{Alexis}{Duque}           % Your name
\lastupdated

%----------------------------------------------------------------------------------------
%	SIDEBAR SECTION  -- In the aside, each new line forces a line break
%----------------------------------------------------------------------------------------

\begin{aside}
%
\section{Contact}
64 rue Lucette et René Desgrand
6100, Villeurbanne
France
~
+33 6 51 24 37 76
~
alexisd61@gmail.com
alexisd@rtone.fr
alexis.duque@insa-lyon.fr
%
\section{Langues}
Anglais, professionnel
{\color{red} $\varheartsuit$} Espagnol, bilingue
Mandarin, notions
%
\section{Programmation}
{\color{red} $\varheartsuit$} C/C++, Java, JS
Android SDK and NDK
Objective C, Swift
Embedded Linux
STM32 MCU
ARM Cortex M0+, M3
Contiki, RIOT
Angular, Node, Cordova
PHP, Symfony2/3, Laravel
SQL, Bash, Python
Docker, Jenkins, Agile
Git, \LaTeX{}
%
\section{Réseaux}
{\color{red} $\varheartsuit$} VLC - 802.15.7
{\color{red} $\varheartsuit$} Bluetooth LE
802.15.4, 6LoWPAN
LPWAN, LoRa, Sigfox
{\color{red} $\varheartsuit$} Sécurité, Forensics
\end{aside}

%  \vspace{-0.2cm}

%----------------------------------------------------------------------------------------

%----------------------------------------------------------------------------------------
%	WORK EXPERIENCE SECTION
%----------------------------------------------------------------------------------------
 \vspace{0.15cm}

\section{Expériences Professionnelles}

\begin{entrylist}
\entry
  {Depuis 2017}
  {Rtone}
  {Lyon, France}
  {\jobtitle{Responsable R\&D et Sécurité}\\
  Responsable technique de l'offre de conseil et audit de sécurité pour l'Internet des Objets.\\
  Gestion de projets collaboratif : SME Instrument, Eurostars, ANR, FUI PACLIDO (Protocoles et Algo. Crypto. pour l'IoT)\\
}
%------------------------------------------------
\entry
  {2015-2018}
  {Rtone}
  {Lyon, France}
  {\jobtitle{Doctorant}\\
  Sujet de thèse : Bidirectional Visible Light Communications for the Internet of Things\\ 
  Développement d'objets communicants grâce à la lumière visible et un smartphone\\
  Redaction et dépôt d'un brevet à l'INPI et l'EPO (European Patent Office)\\
 }
%------------------------------------------------
\entry
  {2014-2017}
  {Rtone}
  {Lyon, France}
  {\jobtitle{Ingénieur R\&D Systèmes Embarqué}\\
  Programmation sur micro-contrôleurs:
  \begin{itemize}
    \item Développement d'un Body-Area-Network pour un contexte sportif (802.15.4)
    \item Mode Low Power et optimisation de la consommation énergétique
    \item STM32. ARM Cortex M0+, M0, M3. TI MSP430
  \end{itemize}
  Développement de firmwares Bluetooth Low Energy sur SoC Nordic et Bluegiga.\\
  Conception d'une passerelle 3G Linux Embarqué: C++, Projet Yocto\\
  Applications mobiles :
  \begin{itemize}
    \item Expertise sur Bluetooth Low Energy
    \item Communication par Ultrasons
    \item iOS 8 en Objective C ou Swift
    \item Android SDK et NDK
    \item Cross plateforme avec Apache Cordova, Ionic, AngularJS
  \end{itemize}
  Applications Web :
  \begin{itemize}
    \item Évolutions et améliorations de plusieurs applications web Java dans le milieu de l'Internet des Objets (Spring, Hibernate, GWT, OSGI)
  \end{itemize}}
%------------------------------------------------
 \entry
  {S1 2015 }
  {ENTEL, Universidad Politecnica de Cataluna}
  {Barcelone, Espagne}
  {\jobtitle{Light sensor development for the Ara platform}\\
  Projet de recherche au sein de l'équipe Wireless Network Group, 5 mois :
  \begin{itemize}
    \item Etat de l'art et veille sur Visible Light Communications (VLC)
    \item Etude de la plate-forme de développement du smartphone modulaire Ara
    \item Conception du circuit électronique d'un récepteur pour la communication par modulation lumineuse. Développement C et Android (SDK et NDK)
	\item Rédaction d'un mémoire de recherche
  \end{itemize}}
%------------------------------------------------
\entry
  {Depuis 2013}
  {Développeur Freelance}
  {Lyon, France}
  {\jobtitle{Développeur Web et Mobile}\\
  Développement d'applications web et mobiles en utilisant des technologies comme Java, Swift, Symphony2/3, Angular, Polymer\\
  Exemple de projets:
  \begin{itemize}
    \item Développement d'une application Android d'aide à conduite de réunion \\ (Laboratoire TechCico , Université Technologique de Troyes)
    \item Conception de l’outil de reporting énergétique de l'usine Michelin Troyes
    \item Conception et développement d'une applications web de dimensionnement de systèmes de protection incendie en suivant les norme NF, NFPA et APSADR1.
    \item Développement d'un système de dossier médical électronique basé sur OpenMRS et Bahmni
  \end{itemize}}
%------------------------------------------------
\end{entrylist}

%----------------------------------------------------------------------------------------
%	EDUCATION SECTION
%----------------------------------------------------------------------------------------
\pagebreak
\section{Formation}

\begin{entrylist}
%------------------------------------------------
\entry
{2015--2018}
{PhD. {\normalfont Bidirectional Visible Light Communications for the Internet of Things}}
{CITI Lab}
{\bodyfontit{Equipe INRIA AGORA, contrat CIFRE avec Rtone}}
%------------------------------------------------
\entry
{2009--2015}
{Ing. {\normalfont mention Télécommunications, Services et Usages.}}
{INSA de Lyon}
{\bodyfontit{Semestre ERASMUS à l'UPC-ETSETB, Barcelone, Espagne} \\
\bodyfontit{2 ans en section internationale Asie. Stage ouvrier de 2 mois à Shanghai, Chine}}
%------------------------------------------------
\end{entrylist}

%----------------------------------------------------------------------------------------
%	OTHER QUALIFICATIONS SECTION
%----------------------------------------------------------------------------------------

%\section{other qualifications}

%\begin{entrylist}
%------------------------------------------------
%\entry
%{2013}
%{Qualification}
%{Institution}
%{\vspace{-0.3cm}}
%------------------------------------------------
%\entry
%{2011}
%{Qualification}
%{Institution}
%{\vspace{-0.3cm}}
%------------------------------------------------
%\end{entrylist}

%----------------------------------------------------------------------------------------
%	AWARDS SECTION
%----------------------------------------------------------------------------------------

%\section{awards}

%\begin{entrylist}
%------------------------------------------------
%\entry
%{2014}
%{Google Summer of Code Student}
%{OpenMRS}
%{Open Source contributor at OpenMRS in the framework of Google Summer of Code fellowship. }
%------------------------------------------------
%\end{entrylist}
%----------------------------------------------------------------------------------------
%	PUBLICATION SECTION
\needspace{5\baselineskip}
\section{Publications}
 \vspace{-0.2cm}
\large{\textbf{Articles dans des conférences académiques}}

\normalsize
\begin{publist}
%------------------------------------------------
\pub
{A. Duque, R. Stanica, H. Rivano, C. Goursaud, and A. Desportes}
{“Poster: Insights into RGB-LED to Smartphone Communication”}
{Proceedings of the 2018 International Conference on Embedded Wireless Systems and Networks}
{EWSN’18, (Madrid, Spain)}\\
\pub
{A. Duque, R. Stanica, H. Rivano, and A. Desportes}
{“Decoding methods in LED-to-smartphone bidirectional communication for the IoT”}
{Proceedings of the 2018 Global LIFI Congress (GLC)}
{GLC’18, (Paris, France)}\\
\pub
{A. Duque, R. Stanica, H. Rivano, and A. Desportes}
{“SeedLight: Hardening LED-to-Camera Communication with Random Linear Coding”}
{Proceedings of the 4th Workshop on Visible Light Communication System}
{VLCS’17, (Snowbird, Utah, USA)}\\
\pub
{A. Duque, R. Stanica, H. Rivano, and A. Desportes}
{“Demo: Off-the-shelf bi-directional visible light communication module for IoT devices and smartphones”}
{Proceedings of the 2017 International Conference on Embedded Wireless Systems and Networks}
{EWSN’17, (Uppsala, Sweden)}\\
\pub
  {A. Duque, R. Stanica, H. Rivano, and A. Desportes}
  {“Unleashing the power of led- to-camera communications for iot devices”}
  {Proceedings of the 3rd Workshop on Visible Light Communication System}
  {VLCS’16, (New York, NY, USA)}\\
\pub
  {A. Duque, R. Stanica, H. Rivano, and A. Desportes} {“Demo: Unleashing the power of led- to-camera communications for iot devices”} {Proceedings of the 3rd Workshop on Visible Light Communication Systems} {VLCS’16, (New York, NY, USA)}
\end{publist}

%------------------------------------------------

\large{\textbf{Brevets}}

\normalsize
\begin{publist}
%------------------------------------------------
\pat
{A. Duque, A. Desportes,  R. Stanica, H. Rivano}
{2017}
{"Procédés de communication en lumière visible"}
{Brevet Européen N° EP18157382}
{Date de Priorité : 17 février 2017}
\\
\end{publist}

% \large{\textbf{Présentations dans des conférences IT}}

% \normalsize
% \begin{publist}
% %------------------------------------------------
% \prez
% {A. Duque, A. Desportes,  R. Stanica, H. Rivano}
% {2017}
% {"Procédés de communication en lumière visible"}
% {Brevet Européen N° : EP18157382}
% {Date de Priorité : 17 février 2017}
% \\
% \end{publist}
%----------------------------------------------------------------------------------------
%	INTERESTS SECTION
%----------------------------------------------------------------------------------------

\section{Centres d'interêt}
  \vspace{-0.2cm}
  
\textbf{Professionnel :} veille technologique,  contributeur open-source sur le projet OpenMRS, participation au Google Summer of Code 2014 et 2016 \\
\textbf{Sport :} duathlon, cyclisme, athlétisme (niveau national) \\
\textbf{Culture :} lecture, pratique de la guitare
%----------------------------------------------------------------------------------------
\end{document}