% -- Encoding UTF-8 without BOM
% -- XeLaTeX => PDF (BIBER)

\documentclass[]{cv-style}          % Add 'print' as an option into the square bracket to remove colours from this template for printing.
                                    % Add 'espanol' as an option into the square bracket to change the date format of the Last Updated Text

\sethyphenation[variant=british]{english}{} % Add words between the {} to avoid them to be cut
\usepackage{etoolbox}

\patchcmd{\header}
{anchor=center}
{anchor=center,text width=\textwidth,align=center}
{}{}

\patchcmd{\header}
{minimum height=2cm}
{minimum height=2.4cm}
{}{}

\patchcmd{\header}
{{\thinfont #1}{\bodyfont  #2}}
{{\thinfont #1}{\bodyfont  #2}\\[6pt]{\bodyfont\large PhD - Director of Research}}
{}{}

\apptocmd{\header}{\vskip2\parskip}{}{}

\patchcmd{\aside}{(1, 1.87)}{(1, 2.4)}{}{}

\begin{document}

\header{Alexis}{Duque}           % Your name
\lastupdated

%----------------------------------------------------------------------------------------
%	SIDEBAR SECTION  -- In the aside, each new line forces a line break
%----------------------------------------------------------------------------------------

\begin{aside}
%
\section{contact}
64 rue Lucette et René Desgrand
69100, Villeurbanne
France
~
+33 6 51 24 37 76
~
alexisd61@gmail.com
alexis.duque@rtone.fr
% alexis.duque@insa-lyon.fr
%
\section{languages}
French mother tongue
English fluency
{\color{red} $\varheartsuit$} Spanish bilingual
Mandarin basic
%
\section{research interests}
{\color{red} $\varheartsuit$} Internet of Things
Visible Light Comm.
Machine Learning
{\color{red} $\varheartsuit$} Bluetooth LE
Software Defied Radio
Lightweight Crypto.
{\color{red} $\varheartsuit$} IoT Cybersecurity
Wireless Comm.
%
\section{programming}
{\color{red} $\varheartsuit$} C/C++, Golang
Python, Jupyter, TensorFlow, TensorFlow Lite
Java, Kotlin
Android SDK/NDK
Swift\& iOS SDK
{\color{red} $\varheartsuit$} Embedded Linux \& Android
ARM Cortex M MCUs
Zephyr, Contiki, RIOT
Docker, SysAdmin, Agile
{\color{red} $\varheartsuit$} Git, Gitlab, CI/CD
\LaTeX{}
%
\end{aside}

%----------------------------------------------------------------------------------------
%	PUBLICATION SECTION


%\section{skills}
%  \vspace{-0.2cm}

%----------------------------------------------------------------------------------------

%----------------------------------------------------------------------------------------
%	WORK EXPERIENCE SECTION
%----------------------------------------------------------------------------------------
\vspace{0.15cm}
\section{work experience}

\begin{entrylist}
%------------------------------------------------
\entry
{2017--Now}
{Rtone}
{Lyon, France}
{\jobtitle{Director of Research - Cybersecurity Leader}\\
Research project management: H2020, SME Instrument, Eurostars, ANR\\
Ongoing projects:
\begin{itemize}
    \item FUI PACLIDO (2017-20): Lightweight Cryptography for IoT. Detection of cyberattack on IoT devices using Machine Learning
    \item H2020 SDR4IoT (2020-21): IoT devices authentication and fingerprinting with Software Defined Radio and Machine Learning
    \item ANACONDA (2020-2022): Automation of IoT device cybersecurity assessment
\end{itemize}
IoT Cybersecurity consulting\\
Teaching: Embedded Systems \& IoT, Bluetooth, IoT cybersecurity\\
Speaker in international tech conferences: Devoxx (FR/BE/MA), FOSDEM, MixIT, ...\\
Technical writing
}
%------------------------------------------------
\entry
  {2015--2018}
  {Rtone}
  {Lyon, France}
  {\jobtitle{PhD student}\\
  PhD.: Bidirectional Visible Light Communications for
 the Internet of Things\\
  Working toward bi-directionnal Device-to-Device communication using LED lights and smartphones. European patent granted}
%------------------------------------------------
\entry
  {2014-- 2017}
  {Rtone}
  {Lyon, France}
  {\jobtitle{Embedded Software and Research Engineer}\\
  Micro-controller programming:
  \begin{itemize}
    \item Body-Area-Network development in a sport context (802.15.4)
    \item Low Power management and optimization
    \item STM32, ARM Cortex M0+, M0 and M3
  \end{itemize}
  Bluetooth Low Energy firmware development on Nordic, Bluegiga SoC\\
  Embedded Linux 3G Gateway: C++, Yocto Project\\
  Mobile Applications :
  \begin{itemize}
    \item Bluetooth Low Energy, Ultrasounds
    \item iOS SDK using Objective C and Swift
    \item Android SDK and NDK
    \item Cross-platform using Apache Cordova and Ionic
  \end{itemize}
  Web Applications :
  \begin{itemize}
    \item Evolution and improvement of various Java WebApp (Spring, Hibernate, GWT)
    \item OSGI Bundle development for the Eclipse Kura project using Eclipse Equinox
  \end{itemize}}
%------------------------------------------------
 \entry
  {H1 2015  }
  {ENTEL, Universidad Politecnica de Cataluna}
  {Barcelona, Spain}
  {\jobtitle{Light sensor development for the Ara platform}\\
  Research project within the Wireless Network Group, 5 months :
  \begin{itemize}
    \item Visible Light Communications (VLC) state of the art
    \item Development of a VLC front-end for the Google Ara modular smartphone
    \item Hardware design and firmware programming in C
    \item Android SDK and NDK programming
  \end{itemize}}
%------------------------------------------------
\entry
  {2013-- Now}
  {Freelance Developer}
  {Lyon, France}
  {\jobtitle{Mobile and Web applications developer}\\
  Various mobile and web applications development using technologies such as Java, Swift, Symphony2/3, Angular, Polymer\\\
  Most relevant projects:
  \begin{itemize}
    \item Meeting reporting Android application for a knowledge management project (TechCico Laboratory, Université Technologique de Troyes)
    \item Intranet web application enabling energy consumption monitoring in a factory (Michelin Troyes)
    \item Web Application allowing fire protection system scaling and planning
    \item Electronic medical recording system development based on OpenMRS and Bahmni
  \end{itemize}}
%------------------------------------------------
\end{entrylist}

%----------------------------------------------------------------------------------------
%	EDUCATION SECTION
%----------------------------------------------------------------------------------------
%\pagebreak
\section{education}

\begin{entrylist}
%------------------------------------------------
\entry
{2015--2018}
{PhD. {\normalfont Bidirectional Visible Light Communications for the Internet of Things}}
{CITI Lab}
{\bodyfontit{INRIA AGORA Team, funded by Rtone (French CIFRE fellowship)}}
%------------------------------------------------
\entry
{2009--2015}
{M.Eng. {\normalfont in Computer Science and Telecommunications.}}
{INSA de Lyon}
{\bodyfontit{Fifth year final semester abroad at the UPC-ETSETB, Spain} \\
\bodyfontit{First two years in Asian international section}}
%------------------------------------------------
\end{entrylist}

%----------------------------------------------------------------------------------------
%	PUBLICATION SECTION
\needspace{5\baselineskip}
\section{publications}
 \vspace{-0.2cm}
\large{\textbf{Journals and international conferences proceedings}}

\normalsize
\begin{publist}
%------------------------------------------------
\pub
{A. Duque, A. Lahmadi, N. Heraief, and J. Francq}
{“MitM Attack Detection in BLE Networks using Reconstruction and Classification Machine Learning Techniques”}
{Proceedings of the 2nd Workshop on Machine Learning for Cybersecurity}
{MLCS’20, (Ghent, Belgium)}\\
\pub
{A. Duque, R. Stanica, H. Rivano, and A. Desportes}
{“Analytical and simulation tools for optical camera communications”}
{Elsevier Computer Communications}
{Vol. 120, pp. 52-62, July 2020}\\
\pub
{A. Duque, R. Stanica, H. Rivano, and A. Desportes}
{“Performance Evaluation of LED-to-Camera Communications”}
{Proceedings of the 22nd ACM International Conference on Modeling, Analysis and Simulation of Wireless and Mobile Systems}
{MSWiM’19, (Miami Beach, FL, USA)}\\
\pub
{A. Duque, R. Stanica, H. Rivano, C. Goursaud, and A. Desportes}
{“Poster: Insights into RGB-LED to Smartphone Communication”}
{Proceedings of the 2018 International Conference on Embedded Wireless Systems and Networks}
{EWSN’18, (Madrid, Spain)}\\
\pub
{A. Duque, R. Stanica, H. Rivano, and A. Desportes}
{“Decoding methods in LED-to-smartphone bidirectional communication for the IoT”}
{Proceedings of the 2018 Global LIFI Congress (GLC)}
{GLC’18, (Paris, France)}\\
\pub
{A. Duque, R. Stanica, H. Rivano, and A. Desportes}
{“SeedLight: Hardening LED-to-Camera Communication with Random Linear Coding”}
{Proceedings of the 4th Workshop on Visible Light Communication System}
{VLCS’17, (Snowbird, Utah, USA)}\\
\pub
{A. Duque, R. Stanica, H. Rivano, and A. Desportes}
{“Demo : Off-the-shelf bi-directional visible light communication module for IoT devices and smartphones”}
{Proceedings of the 2017 International Conference on Embedded Wireless Systems and Networks}
{EWSN’17, (Uppsala, Sweden), 2017}\\
\pub
  {A. Duque, R. Stanica, H. Rivano, and A. Desportes}
  {“Unleashing the power of led- to-camera communications for iot devices”}
  {Proceedings of the 3rd Workshop on Visible Light Communication System}
  {VLCS’16, (New York, NY, USA)}\\
\pub
  {A. Duque, R. Stanica, H. Rivano, and A. Desportes} {“Demo: Unleashing the power of led- to-camera communications for iot devices”} {Proceedings of the 3rd Workshop on Visible Light Communication Systems} {VLCS’16, (New York, NY, USA)}
\end{publist}

\large{\textbf{Patents}}

\normalsize
\begin{publist}
%------------------------------------------------
\pat
{A. Duque, A. Desportes,  R. Stanica, H. Rivano}
{2017}
{"Procédés de communication en lumière visible"}
{European Patent N° EP18157382}
{Priority Date: February 17th 2017}
\\
\end{publist}
\newpage
\large{\textbf{Technical press}}

\normalsize
\begin{publist}
%------------------------------------------------
\pub
  {A. Duque} {“TensorFlow Lite im Vertical Farming”} {Linux Magazin } {April 2020}\\
\pub
  {A. Duque} {“Run TensorFlow models on edge devices”} {ADMIN Magazine} {Issue 57, May 2020}\\
\pub
  {A. Duque} {“Machine Learning sur des objets connectés avec TensorFlow Lite pour l’agriculture verticale”} {Linux Magazine France} {Issue 236, July 2020}
  
%------------------------------------------------
\end{publist}

%------------------------------------------------

%----------------------------------------------------------------------------------------
%	OTHER QUALIFICATIONS SECTION
%----------------------------------------------------------------------------------------

%\section{other qualifications}

%\begin{entrylist}
%------------------------------------------------
%\entry
%{2013}
%{Qualification}
%{Institution}
%{\vspace{-0.3cm}}
%------------------------------------------------
%\entry
%{2011}
%{Qualification}
%{Institution}
%{\vspace{-0.3cm}}
%------------------------------------------------
%\end{entrylist}

%----------------------------------------------------------------------------------------
%	AWARDS SECTION
%----------------------------------------------------------------------------------------

\section{certifications \& awards}

\begin{entrylist}
%------------------------------------------------
\entry
{June 2020}
{TensorFlow Developer Certificate}
{Google}
{ID 8TGQMHD85RS7 \\ Integrating machine learning into tools and applications. Understanding of building TensorFlow models using Computer Vision, Convolutional Neural Networks, Natural Language Processing, and real-world image data and strategies.}
%------------------------------------------------
\end{entrylist}

%----------------------------------------------------------------------------------------
%	INTERESTS SECTION
%----------------------------------------------------------------------------------------

\section{interests}
  \vspace{-0.2cm}

\textbf{professional :} technological watch,  free and open-source software activist, contributor at OpenMRS, Google Summer of Code 2014 and 2016 student\\
\textbf{sport :} triathlon, duathlon, cycling, athletics \\
\textbf{culture :} reading, practicing the classical and electric guitar
%----------------------------------------------------------------------------------------
\end{document}